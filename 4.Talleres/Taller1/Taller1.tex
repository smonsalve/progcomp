\documentclass[11pt,letterpaper]{article}
\usepackage{samstyle}

\title{
	Taller 1\\  
	Programación de Computadores\\
	Grupo 063
}

% \author{

% }

\begin{document}
 
\pagestyle{fancyplain}
\fancyhf{}
\headheight=20pt %para cambiar el tamaño del encabezado
\renewcommand{\headrulewidth}{0pt} %espesor del encabezado

% \lhead %la "L" indica a la izquierda
% {
% }

%\fancyfoot[c]{\thepage}

\maketitle

% \begin{minipage}{3cm}
% %\includegraphics[width=15cm]{aux/SamCanny.jpg}
% \end{minipage}

%\begin{frame}{Para hoy...}
	\begin{enumerate}
		\item Asistencia
		\item Monitorias
		\item Quiz
		\item Taller
		\item Dudas Capitulo
		\item Entrada y Salida
		\item Arreglos
		\item Diccionarios
		\item Listas
		\item Ejercicio
	\end{enumerate}
\end{frame}

\begin{frame}{}
	test\cite{sergiomonsalve}

\end{frame}

% Strings
% Rangos
% Estructuras de Control
% Ciclos
% 	For Loops
% 	while

% \newpage

% \newpage
\begin{center}
\begin{tabular}{c c}
	Profesor & Monitor \\
	Sergio Andrés Monsalve Castañeda & Marcos David Sierra Gallego\\
	smonsal3@eafit.edu.co & msierr37@eafit.edu.co
\end{tabular}
\end{center}
\vspace{1cm} 

Para el primer taller de la semana es necesario que lleven resueltos los siguientes ejercicios de CodingBat\cite{Codingbat}, de los cuales les entregamos una rápida traducción.

\begin{enumerate}
	\item Python \textgreater Warmup-2 \textgreater front\_times:\\
		Dada una cadena y un entero no negativo (n), diremos que el frente de la cadena son los 3 primeros caracteres o lo que sea que haya en caso de que la longitud de la cadena sea menor a 3, retorne 3 copias del frente de la cadena. 

	\item Python \textgreater Warmup-1 \textgreater sum\_double:\\
		Dados dos números (a y b) retorne su suma, en caso de que a y b sean iguales retorne el doble de la suma de ambos.

	\item Python \textgreater String-2 \textgreater xyz\_there:\\
		Retorne verdadero si la cadena de caracteres dada contiene una aparición de \"xyz\", donde xyz no esta inmediatamente precedida de un punto (.) por lo que \"xxyz\" es valido pero \"x.xyz\" no.
	
	\item Python \textgreater String-1 \textgreater non\_start:\\
		Dadas 2 cadenas(a,b), retorne la concatenación de ambas (a+b), excepto el primer caracter de ambas, las cadenas tendrán longitud mayor o igual a 1.

	\item Python \textgreater String-1 \textgreater combo\_string:\\
		Dadas 2 cadenas (a,b) retorne una cadena de la forma corta+larga+corta, con la cadena corta en la parte de afuera de la cadena resultante y la larga adentro. Las cadenas nunca serán del mismo tamaño, mas pueden ser de longitud 0. 
\end{enumerate}


\bibliographystyle{IEEEtran}
\bibliography{Refs}

\end{document}