\documentclass[11pt,twoside,a4paper]{article}
\usepackage{samstyle}

\title{
	Taller 2\\
 	Programación de Computadores
}

\author{Sergio Andrés Monsalve Castañeda}

\begin{document}
\maketitle

	\section{Contenido}
		Se recomienda estudiar y tener claro los siguientes temas: 
		\begin{itemize}
			\item Entrada y Salida
			\item Excepciones
			\item Funciones Recursivas
			\item Estructuras de Datos (Listas, Tuplas, Conjuntos, Diccionarios)
			\item Busqueda
			\item Ordenamiento (Quicksort)
		\end{itemize}

	\newpage
	\section{Ejercicios}

	\begin{enumerate}

		\item Dado un archivo que contenga en la primera linea, un entero N que indique la cantidad de lineas que le preceden, poder procesar cada uno de los elementos
		\lstinputlisting{file1.txt}
		\item Dado un archivo que contenga N lineas de codigo con datos a procesar, poder procesar cada uno de los elementos.
		\lstinputlisting{file2.txt}

		\item Dada una lista de números enteros, escribir una función que:

			\begin{itemize}

			    \item Devuelva una lista con todos los que sean primos.

			    \item Devuelva la sumatoria y el promedio de los valores.

			    \item Devuelva una lista con el factorial de cada uno de esos números.

			\end{itemize}

		\item Dada una lista de números enteros y un entero k, escribir una función que devuelva tres listas, una con los menores, otra con los mayores y otra con los iguales a k.

		\item Escribir una función empaquetar para una lista, donde epaquetar significa indicar la repetición de valores consecutivos mediante listas auxiliares (valor, cantidad de repeticiones). Por ejemplo, empaquetar [1, 1, 1, 3, 5, 1, 1, 3, 3] debe devolver [[1, 3] , [3, 1] , [5, 1], [1, 2], [3, 2]].
		\newpage
		\item Escribir un programa que reciba una palabra y un archivo e imprima una diccionario con las palabras existentes en el archivo y por cada palabra una lista con las lineas en las que aparece tal palabra
		donde un archivo como:
		\lstinputlisting{dict.txt}
		daría por salida: 
		\lstinputlisting{dict.o}

	\end{enumerate}

	Algunos ejercicios fueron tomados de \cite{Ejerc68:online} otros son modificaciones.

\bibliographystyle{IEEEtran}
\bibliography{refs.bib}


\end{document}
