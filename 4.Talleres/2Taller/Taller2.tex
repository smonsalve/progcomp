\documentclass[11pt,twoside,a4paper]{article}
\usepackage{samstyle}

\begin{document}

	\begin{enumerate}

		\item Dada una lista de números enteros, escribir una función que:

			\begin{itemize}

			    \item Devuelva una lista con todos los que sean primos.

			    \item Devuelva la sumatoria y el promedio de los valores.

			    \item Devuelva una lista con el factorial de cada uno de esos números.

			\end{itemize}

		\item Dada una lista de números enteros y un entero k, escribir una función que:

			\begin{itemize}

				\item   Devuelva tres listas, una con los menores, otra con los mayores y otra con los iguales a k.

				\item   Devuelva una lista con aquellos que son múltiplos de k.
				
			\end{itemize}

		\item Realizar una función que, dada una lista, devuelva una nueva lista cuyo contenido sea igual a la original pero invertida. Así, dada la lista [’Di’, ’buen’, ’día’, ’a’, ’papa’], deberá devolver [’papa’, ’a’, ’día’, ’buen’, ’Di’].

		\item Realizar otra función que invierta la lista, pero en lugar de devolver una nueva, modifique la lista dada para invertirla, usar listas auxiliares.

		\item Escribir una función empaquetar para una lista, donde epaquetar significa indicar la repetición de valores consecutivos mediante listas auxiliares (valor, cantidad de repeticiones). Por ejemplo, empaquetar [1, 1, 1, 3, 5, 1, 1, 3, 3] debe devolver [[1, 3] , [3, 1] , [5, 1], [1, 2], [3, 2]].

		\item Escribir una función que reciba dos matrices y devuelva la suma.(Investigar suma de matrices)

		\item Escribir una función que reciba dos matrices y devuelva el producto.(Investigar multiplicación de matrices)

		\item Escribir un programa, que reciba un archivo, lo procese e imprima por pantalla cuantas palabras contiene el archivo(Ayuda: Contar espacios y saltos de linea)

		\item Escribir un programa que reciba una palabra y un archivo e imprima las líneas del archivo que contienen la palabra recibida.

	\end{enumerate}

\end{document}
